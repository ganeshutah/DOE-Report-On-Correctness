The exercise of bringing this limited set of report authors together for sharing ideas has resulted in good cross fertilization of ideas for HPC correctness: making us aware of useful tools for our own research in HPC, and some of the larger challenges.  But with the breadth of the problem, and the richness of the verification and debugging community outside, HPC, more is needed.

Advancement in this area of correctness could be facilitated by a workshop on HPC correctness that could bring together the larger community of experts on correctness techniques and tools with DOE stakeholders, especially the developers of DOE HPC applications. 

Such a workshop would provide an opportunity for HPC software developers to communicate current practices and discuss the primary obstacles to achieving correctness in HPC software development and for correctness experts to identify promising research directions that offer the potential to overcoming these obstacles.  A two day workshop comprising a small number of invited presentations, 5--10 minute presentations based on 2-page position papers solicited from the community, and 3--4 1-hour round table discussions to stimulate a dialogue between the correctness experts and stakeholders is recommended.  We anticipate that this dialogue will reinforce many of the summit findings, possibly identify additional research opportunities, and help prioritize future research directions.
%--
