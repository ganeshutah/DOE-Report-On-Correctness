\begin{WrapTextLeft}
\footnotesize
{\large \textbf{When More Than \textit{print} Debugging Is Needed}}\\
A scientist experienced hangs in a laser-plasma interaction code 
(named PF3D) when scaling it to 524,288 MPI processes on 
LLNL’s Sequoia BlueGene/Q system. The scientist spent months trying 
to debug the problem through print statements to no avail. 
Moreover, the scientist was unable to reproduce the hang at 
smaller scales where fully featured, heavyweight debuggers 
would be more plausible. Using STAT (Stack Trace Analysis Tool), 
the scientist was able to debug the problem, a race condition 
between two distinct but overlapping communication code regions. 
The bug was the result of the application migrating from one version to another more scalable but incompatible one. During migration, the application ran 
through a compatibility layer that introduced the race condition 
and ultimately caused the timing- and scale-dependent hangs 
(more in~\cite{CACM:Debugging}). This case shows that, although 
print debugging can be a useful debugging method, 
the HPC community can benefit from advanced correctness 
methods and tools to isolate bugs that otherwise can consume 
months of effort and millions of CPU hours to fix.

\end{WrapTextLeft}